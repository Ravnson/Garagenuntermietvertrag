\documentclass[a4paper]{scrartcl}
\usepackage[contract]{scrjura}

% for checkboxes
\usepackage{pifont}
\usepackage{amssymb}

\usepackage{multicol}

\title{Garagenuntermietvertrag}
% no date on title
\date{}

% For autofill, use it at your wish
\newcommand{\VermieterName}{Max Mustermann}
\newcommand{\VermieterStrasse}{Musterstraße 42}
\newcommand{\VermieterOrt}{12345 Musterstadt}
\newcommand{\MietsacheStrasse}{Vorlagenallee 17}
\newcommand{\MietsacheOrt}{54321 Vorlagenstadt}
\newcommand{\Miete}{96}

\begin{document}
	
\maketitle

\paragraph{Zwischen} 

(im folgenden "Untervermieter" genannt) \\

\begin{tabular}{p{0.2\linewidth}p{0.7\linewidth}}
	Vor-/Nachname:&\hrulefill
	
	\\
	
	Straße/Haus-Nr.:&\hrulefill
	
	\\
	
	PLZ/Ort:&\hrulefill
\end{tabular}\\

\paragraph{und}

(im folgenden "Untermieter" genannt) \\

\begin{tabular}{p{0.2\linewidth}p{0.7\linewidth}}
	Vor-/Nachname:&\hrulefill
	
	\\
	
	Straße/Haus-Nr.:&\hrulefill
	
	\\
	
	PLZ/Ort:&\hrulefill
\end{tabular}\\

\vspace{5mm}

wird folgender \textbf{Garagenuntermietvertrag} geschlossen:


\Clause{title=Mietsache}

Der Untervermieter vermietet dem Untermieter zum Abstellen
\vspace{3mm} \\
$\square$ eines PKW \\
$\square$ eines Motorrads \\
$\square$ des folgenden Fahrzeugs \hrulefill 
\vspace{3mm}\\
die Garage auf dem Grundstück
\vspace{3mm} \\
\begin{tabular}{p{0.2\linewidth}p{0.7\linewidth}}
	Straße/Haus-Nr.:&\hrulefill
	
	\\
	
	PLZ/Ort:&\hrulefill
\end{tabular}\\

\Clause{title=Mietzeit}

Das Mietverhältnis beginnt am $\rule{4cm}{0.15mm}$ \vspace{3mm}

\noindent
Das Mietverhältnis läuft auf \vspace{2mm}\\
$\square$ unbestimmte Zeit\\
$\square$ bestimmte Zeit, und zwar bis zum $\rule{4cm}{0.15mm}$


\Clause{title=Miete} 

Die Miete beträgt monatlich $\rule{2cm}{0.15mm}$ \ Euro einschließlich Nebenkosten. \vspace{3mm}

\noindent
Die Miete ist monatlich spätestens am dritten Werktag eines jeden Monats im Voraus an 
den Untervermieter zu entrichten. Die Zahlungen sind auf das folgende Konto zu überweisen:
\vspace{3mm}\\
\begin{tabular}{p{0.2\linewidth}p{0.7\linewidth}}
	Kontoinhaber:&\hrulefill
	
	\\
	
	Kreditintitut:&\hrulefill
	
	\\
	
	IBAN:&\hrulefill
\end{tabular}\vspace{3mm}


\noindent
Ausschlaggebender Zeitpunkt ist die Wertstellung auf dem Konto des Vermieters, nicht der Tag der Absendung.

\Clause{title=Kündigung}

Das Mietverhältnis kann von jeder Partei, insbesondere dem Vermieter, spätestens am dritten Werktag eines Kalendermonats zum Ende des übernächsten Kalendermonats gekündigt werden.
\vspace{3mm}

\noindent
Der Untervermieter ist zur fristlosen Kündigung berechtigt, wenn der Untermieter mit einem Betrag, der zwei Monatsmieten
übersteigt, im Rückstand ist oder wenn der Untermieter das Objekt vertragswidrig benutzt. Hierzu zählt auch die 
Überlassung an Dritte. 
\vspace{3mm}

\noindent
Der Untermieter ist zur fristlosen Kündigung berechtigt, wenn ihm der Gebrauch der Garage in erheblichem Maße 
nicht gewährt oder wieder entzogen wird.


\Clause{title=Schlüssel}

Der Untermieter erhält $\rule{2cm}{.15mm}$ Schlüssel.  Von  dem/den  erhaltenen Schlüssel/n  darf  der  Untermieter  ohne Zustimmung  des  Vermieters  keine weiteren Schlüssel anfertigen lassen. Der Verlust eines oder mehrerer 
Schlüssel ist dem Untervermieter unverzüglich mitzuteilen. Kosten,  die  durch  den  Verlust  des  Schlüssels  entstehen  
(z.B.  eventuell  notwendiger  Austausch  des  Türschlosses  oder  Ersatz-beschaffung des Schlüssels), trägt 
der Untermieter, sofern er den Verlust zu verantworten hat. 


\Clause{title=Benutzung der Mietsache}

Eine andere Nutzung der Mietsache als zu den in § 1 bestimmten Zwecken, ist nur mit vorheriger schriftlicher 
Zustimmung des Vermieters gestattet.
\vspace{3mm}

\noindent
Das  zum  Betrieb,  zur  Wartung  und  zur  Pflege  des  Fahrzeugs  erforderliche  Zubehör  (Reifen,  
Werkzeug,  Putzmittel)  darf  der  Untermieter in der Garage ohne Zustimmung des Vermieters aufbewahren. 
\vspace{3mm}

\noindent
Der Zugang zur Garage darf nur im Schritttempo befahren werden
\vspace{3mm}

\noindent
Der Untermieter verpflichtet sich, das Fahrzeug weder in der Garage noch auf dem Grundstück zu waschen oder
zu reparieren. Des weiteren verpflichtet er sich elektrischen Strom nur zu Beleuchtungszwecken zu verwenden.
\vspace{3mm}


\noindent
Bei der Benutzung der Garage ist der Untermieter verpflichtet, die einschlägigen sicherheitsrechtlichen Vorschriften 
über die Lagerung von brennbaren Gegenständen zu beachten. Insbesondere verpflichtet sich der Untervermieter,  die Garage nicht mit offenem Licht oder Feuer zu betreten, keine Betriebsstoffe oder feuergefährliche Gegenstände in der
Garage zu lagern sowie den Motor nicht bei geschlossener Garage laufen zu lassen.

\Clause{title=Instandhaltung der Mietsache}

Der Untermieter hat für die Reinigung und Sauberhaltung der Garage zu sorgen.
\vspace{3mm}

\noindent
Zeigt sich ein wesentlicher Mangel der Mietsache oder wird eine Vorkehrung zum Schutze der Mietsache 
gegen eine nicht vorher gesehene Gefahr erforderlich, so hat der Untermieter dies dem Untervermieter unverzüglich zu melden. 

\Clause{title=Schnee- und Eisbeseitigung}

Die Beseitigung von Schnee und Eis von den Zufahrten zur Mietsache wird von der Hausverwaltung durchgeführt.


\Clause{title=Überlassung der Mietsache an Dritte}

Die Garage darf ohne Zustimmung des Vermieters kurzfristig unentgeltlich an Dritte zur Verfügung gestellt werden. Für etwaige Schäden, die durch die Benutzung dieser Personen an der Mietsache entstehen, haftet der Untermieter.
\vspace{3mm}

\noindent
Weitere Untervermietung durch den Untermieter an Dritte ist nicht gestattet.


\Clause{title=Haftung des Untermieters}

Der Untermieter haftet für die Verunreinigungen durch Öl oder Benzin sowie für alle Schäden, die bei der Benutzung
der Garage oder infolge der Nichtbeachtung vertraglicher oder gesetzlicher Vorschriften durch ihn selbst oder durch 
andere Personen, denen er die Benutzung seines Kraftfahrzeugs bzw. der Garage gestattet hat, schuldhaft verursacht 
werden.


\Clause{title=Haftung des Untervermieters}

Der Untervermieter hat auftretende bzw. bestehende Mängel an der Mietsache selbst dem Mietvertrag entsprechend
zu behandeln, sofern diese Mängel nicht durch den Mieter bzw. durch Personen, denen der Mieter die Benutzung der
Garage gestattet hat, schuldhaft verursacht wurden und der und der Untervermieter vom Untermieter auf diese Mängel
hingewiesen wurde (vgl. § 7 Abs. 2 ).


\Clause{title=Beendigung des Untermietverhältnisses}

Bei Beendigung des Untermietverhältnisses hat der Mieter die Garage vollständig geräumt, gereinigt und mit sämtlichen,
ihm überlassenen und von ihm zusätzlich beschafften Schlüsseln zurück zu geben.


\Clause{title=Zusätzliche Vereinbarungen}

Die Parteien des Untermietvertrages vereinbaren zusätzlich zu den vorgenannten Vorschriften noch folgende
Vereinbarungen: \vspace{5cm}


\Clause{title=Nebenabreden}

Mündliche Abreden bestehen nicht. Änderungen dieses Vertrages bedürfen der Schriftform.

\vspace{4cm}

\parnumberfalse
\noindent
\begin{tabular}{p{0.3\linewidth}p{0.3\linewidth}p{0.3\linewidth}}
	\hrulefill&\hrulefill&\hrulefill\\
	\scriptsize Datum&{\scriptsize Unterschrift des Untervermieters}&{\scriptsize Unterschrift des Untermieters}
\end{tabular}


	
\end{document}